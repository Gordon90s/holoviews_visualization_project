
% Default to the notebook output style

    


% Inherit from the specified cell style.




    
\documentclass[11pt]{article}

    
    
    \usepackage[T1]{fontenc}
    % Nicer default font (+ math font) than Computer Modern for most use cases
    \usepackage{mathpazo}

    % Basic figure setup, for now with no caption control since it's done
    % automatically by Pandoc (which extracts ![](path) syntax from Markdown).
    \usepackage{graphicx}
    % We will generate all images so they have a width \maxwidth. This means
    % that they will get their normal width if they fit onto the page, but
    % are scaled down if they would overflow the margins.
    \makeatletter
    \def\maxwidth{\ifdim\Gin@nat@width>\linewidth\linewidth
    \else\Gin@nat@width\fi}
    \makeatother
    \let\Oldincludegraphics\includegraphics
    % Set max figure width to be 80% of text width, for now hardcoded.
    \renewcommand{\includegraphics}[1]{\Oldincludegraphics[width=.8\maxwidth]{#1}}
    % Ensure that by default, figures have no caption (until we provide a
    % proper Figure object with a Caption API and a way to capture that
    % in the conversion process - todo).
    \usepackage{caption}
    \DeclareCaptionLabelFormat{nolabel}{}
    \captionsetup{labelformat=nolabel}

    \usepackage{adjustbox} % Used to constrain images to a maximum size 
    \usepackage{xcolor} % Allow colors to be defined
    \usepackage{enumerate} % Needed for markdown enumerations to work
    \usepackage{geometry} % Used to adjust the document margins
    \usepackage{amsmath} % Equations
    \usepackage{amssymb} % Equations
    \usepackage{textcomp} % defines textquotesingle
    % Hack from http://tex.stackexchange.com/a/47451/13684:
    \AtBeginDocument{%
        \def\PYZsq{\textquotesingle}% Upright quotes in Pygmentized code
    }
    \usepackage{upquote} % Upright quotes for verbatim code
    \usepackage{eurosym} % defines \euro
    \usepackage[mathletters]{ucs} % Extended unicode (utf-8) support
    \usepackage[utf8x]{inputenc} % Allow utf-8 characters in the tex document
    \usepackage{fancyvrb} % verbatim replacement that allows latex
    \usepackage{grffile} % extends the file name processing of package graphics 
                         % to support a larger range 
    % The hyperref package gives us a pdf with properly built
    % internal navigation ('pdf bookmarks' for the table of contents,
    % internal cross-reference links, web links for URLs, etc.)
    \usepackage{hyperref}
    \usepackage{longtable} % longtable support required by pandoc >1.10
    \usepackage{booktabs}  % table support for pandoc > 1.12.2
    \usepackage[inline]{enumitem} % IRkernel/repr support (it uses the enumerate* environment)
    \usepackage[normalem]{ulem} % ulem is needed to support strikethroughs (\sout)
                                % normalem makes italics be italics, not underlines
    

    
    
    % Colors for the hyperref package
    \definecolor{urlcolor}{rgb}{0,.145,.698}
    \definecolor{linkcolor}{rgb}{.71,0.21,0.01}
    \definecolor{citecolor}{rgb}{.12,.54,.11}

    % ANSI colors
    \definecolor{ansi-black}{HTML}{3E424D}
    \definecolor{ansi-black-intense}{HTML}{282C36}
    \definecolor{ansi-red}{HTML}{E75C58}
    \definecolor{ansi-red-intense}{HTML}{B22B31}
    \definecolor{ansi-green}{HTML}{00A250}
    \definecolor{ansi-green-intense}{HTML}{007427}
    \definecolor{ansi-yellow}{HTML}{DDB62B}
    \definecolor{ansi-yellow-intense}{HTML}{B27D12}
    \definecolor{ansi-blue}{HTML}{208FFB}
    \definecolor{ansi-blue-intense}{HTML}{0065CA}
    \definecolor{ansi-magenta}{HTML}{D160C4}
    \definecolor{ansi-magenta-intense}{HTML}{A03196}
    \definecolor{ansi-cyan}{HTML}{60C6C8}
    \definecolor{ansi-cyan-intense}{HTML}{258F8F}
    \definecolor{ansi-white}{HTML}{C5C1B4}
    \definecolor{ansi-white-intense}{HTML}{A1A6B2}

    % commands and environments needed by pandoc snippets
    % extracted from the output of `pandoc -s`
    \providecommand{\tightlist}{%
      \setlength{\itemsep}{0pt}\setlength{\parskip}{0pt}}
    \DefineVerbatimEnvironment{Highlighting}{Verbatim}{commandchars=\\\{\}}
    % Add ',fontsize=\small' for more characters per line
    \newenvironment{Shaded}{}{}
    \newcommand{\KeywordTok}[1]{\textcolor[rgb]{0.00,0.44,0.13}{\textbf{{#1}}}}
    \newcommand{\DataTypeTok}[1]{\textcolor[rgb]{0.56,0.13,0.00}{{#1}}}
    \newcommand{\DecValTok}[1]{\textcolor[rgb]{0.25,0.63,0.44}{{#1}}}
    \newcommand{\BaseNTok}[1]{\textcolor[rgb]{0.25,0.63,0.44}{{#1}}}
    \newcommand{\FloatTok}[1]{\textcolor[rgb]{0.25,0.63,0.44}{{#1}}}
    \newcommand{\CharTok}[1]{\textcolor[rgb]{0.25,0.44,0.63}{{#1}}}
    \newcommand{\StringTok}[1]{\textcolor[rgb]{0.25,0.44,0.63}{{#1}}}
    \newcommand{\CommentTok}[1]{\textcolor[rgb]{0.38,0.63,0.69}{\textit{{#1}}}}
    \newcommand{\OtherTok}[1]{\textcolor[rgb]{0.00,0.44,0.13}{{#1}}}
    \newcommand{\AlertTok}[1]{\textcolor[rgb]{1.00,0.00,0.00}{\textbf{{#1}}}}
    \newcommand{\FunctionTok}[1]{\textcolor[rgb]{0.02,0.16,0.49}{{#1}}}
    \newcommand{\RegionMarkerTok}[1]{{#1}}
    \newcommand{\ErrorTok}[1]{\textcolor[rgb]{1.00,0.00,0.00}{\textbf{{#1}}}}
    \newcommand{\NormalTok}[1]{{#1}}
    
    % Additional commands for more recent versions of Pandoc
    \newcommand{\ConstantTok}[1]{\textcolor[rgb]{0.53,0.00,0.00}{{#1}}}
    \newcommand{\SpecialCharTok}[1]{\textcolor[rgb]{0.25,0.44,0.63}{{#1}}}
    \newcommand{\VerbatimStringTok}[1]{\textcolor[rgb]{0.25,0.44,0.63}{{#1}}}
    \newcommand{\SpecialStringTok}[1]{\textcolor[rgb]{0.73,0.40,0.53}{{#1}}}
    \newcommand{\ImportTok}[1]{{#1}}
    \newcommand{\DocumentationTok}[1]{\textcolor[rgb]{0.73,0.13,0.13}{\textit{{#1}}}}
    \newcommand{\AnnotationTok}[1]{\textcolor[rgb]{0.38,0.63,0.69}{\textbf{\textit{{#1}}}}}
    \newcommand{\CommentVarTok}[1]{\textcolor[rgb]{0.38,0.63,0.69}{\textbf{\textit{{#1}}}}}
    \newcommand{\VariableTok}[1]{\textcolor[rgb]{0.10,0.09,0.49}{{#1}}}
    \newcommand{\ControlFlowTok}[1]{\textcolor[rgb]{0.00,0.44,0.13}{\textbf{{#1}}}}
    \newcommand{\OperatorTok}[1]{\textcolor[rgb]{0.40,0.40,0.40}{{#1}}}
    \newcommand{\BuiltInTok}[1]{{#1}}
    \newcommand{\ExtensionTok}[1]{{#1}}
    \newcommand{\PreprocessorTok}[1]{\textcolor[rgb]{0.74,0.48,0.00}{{#1}}}
    \newcommand{\AttributeTok}[1]{\textcolor[rgb]{0.49,0.56,0.16}{{#1}}}
    \newcommand{\InformationTok}[1]{\textcolor[rgb]{0.38,0.63,0.69}{\textbf{\textit{{#1}}}}}
    \newcommand{\WarningTok}[1]{\textcolor[rgb]{0.38,0.63,0.69}{\textbf{\textit{{#1}}}}}
    
    
    % Define a nice break command that doesn't care if a line doesn't already
    % exist.
    \def\br{\hspace*{\fill} \\* }
    % Math Jax compatability definitions
    \def\gt{>}
    \def\lt{<}
    % Document parameters
    \title{Mathematic Visualization - Report}
    
    
    

    % Pygments definitions
    
\makeatletter
\def\PY@reset{\let\PY@it=\relax \let\PY@bf=\relax%
    \let\PY@ul=\relax \let\PY@tc=\relax%
    \let\PY@bc=\relax \let\PY@ff=\relax}
\def\PY@tok#1{\csname PY@tok@#1\endcsname}
\def\PY@toks#1+{\ifx\relax#1\empty\else%
    \PY@tok{#1}\expandafter\PY@toks\fi}
\def\PY@do#1{\PY@bc{\PY@tc{\PY@ul{%
    \PY@it{\PY@bf{\PY@ff{#1}}}}}}}
\def\PY#1#2{\PY@reset\PY@toks#1+\relax+\PY@do{#2}}

\expandafter\def\csname PY@tok@w\endcsname{\def\PY@tc##1{\textcolor[rgb]{0.73,0.73,0.73}{##1}}}
\expandafter\def\csname PY@tok@c\endcsname{\let\PY@it=\textit\def\PY@tc##1{\textcolor[rgb]{0.25,0.50,0.50}{##1}}}
\expandafter\def\csname PY@tok@cp\endcsname{\def\PY@tc##1{\textcolor[rgb]{0.74,0.48,0.00}{##1}}}
\expandafter\def\csname PY@tok@k\endcsname{\let\PY@bf=\textbf\def\PY@tc##1{\textcolor[rgb]{0.00,0.50,0.00}{##1}}}
\expandafter\def\csname PY@tok@kp\endcsname{\def\PY@tc##1{\textcolor[rgb]{0.00,0.50,0.00}{##1}}}
\expandafter\def\csname PY@tok@kt\endcsname{\def\PY@tc##1{\textcolor[rgb]{0.69,0.00,0.25}{##1}}}
\expandafter\def\csname PY@tok@o\endcsname{\def\PY@tc##1{\textcolor[rgb]{0.40,0.40,0.40}{##1}}}
\expandafter\def\csname PY@tok@ow\endcsname{\let\PY@bf=\textbf\def\PY@tc##1{\textcolor[rgb]{0.67,0.13,1.00}{##1}}}
\expandafter\def\csname PY@tok@nb\endcsname{\def\PY@tc##1{\textcolor[rgb]{0.00,0.50,0.00}{##1}}}
\expandafter\def\csname PY@tok@nf\endcsname{\def\PY@tc##1{\textcolor[rgb]{0.00,0.00,1.00}{##1}}}
\expandafter\def\csname PY@tok@nc\endcsname{\let\PY@bf=\textbf\def\PY@tc##1{\textcolor[rgb]{0.00,0.00,1.00}{##1}}}
\expandafter\def\csname PY@tok@nn\endcsname{\let\PY@bf=\textbf\def\PY@tc##1{\textcolor[rgb]{0.00,0.00,1.00}{##1}}}
\expandafter\def\csname PY@tok@ne\endcsname{\let\PY@bf=\textbf\def\PY@tc##1{\textcolor[rgb]{0.82,0.25,0.23}{##1}}}
\expandafter\def\csname PY@tok@nv\endcsname{\def\PY@tc##1{\textcolor[rgb]{0.10,0.09,0.49}{##1}}}
\expandafter\def\csname PY@tok@no\endcsname{\def\PY@tc##1{\textcolor[rgb]{0.53,0.00,0.00}{##1}}}
\expandafter\def\csname PY@tok@nl\endcsname{\def\PY@tc##1{\textcolor[rgb]{0.63,0.63,0.00}{##1}}}
\expandafter\def\csname PY@tok@ni\endcsname{\let\PY@bf=\textbf\def\PY@tc##1{\textcolor[rgb]{0.60,0.60,0.60}{##1}}}
\expandafter\def\csname PY@tok@na\endcsname{\def\PY@tc##1{\textcolor[rgb]{0.49,0.56,0.16}{##1}}}
\expandafter\def\csname PY@tok@nt\endcsname{\let\PY@bf=\textbf\def\PY@tc##1{\textcolor[rgb]{0.00,0.50,0.00}{##1}}}
\expandafter\def\csname PY@tok@nd\endcsname{\def\PY@tc##1{\textcolor[rgb]{0.67,0.13,1.00}{##1}}}
\expandafter\def\csname PY@tok@s\endcsname{\def\PY@tc##1{\textcolor[rgb]{0.73,0.13,0.13}{##1}}}
\expandafter\def\csname PY@tok@sd\endcsname{\let\PY@it=\textit\def\PY@tc##1{\textcolor[rgb]{0.73,0.13,0.13}{##1}}}
\expandafter\def\csname PY@tok@si\endcsname{\let\PY@bf=\textbf\def\PY@tc##1{\textcolor[rgb]{0.73,0.40,0.53}{##1}}}
\expandafter\def\csname PY@tok@se\endcsname{\let\PY@bf=\textbf\def\PY@tc##1{\textcolor[rgb]{0.73,0.40,0.13}{##1}}}
\expandafter\def\csname PY@tok@sr\endcsname{\def\PY@tc##1{\textcolor[rgb]{0.73,0.40,0.53}{##1}}}
\expandafter\def\csname PY@tok@ss\endcsname{\def\PY@tc##1{\textcolor[rgb]{0.10,0.09,0.49}{##1}}}
\expandafter\def\csname PY@tok@sx\endcsname{\def\PY@tc##1{\textcolor[rgb]{0.00,0.50,0.00}{##1}}}
\expandafter\def\csname PY@tok@m\endcsname{\def\PY@tc##1{\textcolor[rgb]{0.40,0.40,0.40}{##1}}}
\expandafter\def\csname PY@tok@gh\endcsname{\let\PY@bf=\textbf\def\PY@tc##1{\textcolor[rgb]{0.00,0.00,0.50}{##1}}}
\expandafter\def\csname PY@tok@gu\endcsname{\let\PY@bf=\textbf\def\PY@tc##1{\textcolor[rgb]{0.50,0.00,0.50}{##1}}}
\expandafter\def\csname PY@tok@gd\endcsname{\def\PY@tc##1{\textcolor[rgb]{0.63,0.00,0.00}{##1}}}
\expandafter\def\csname PY@tok@gi\endcsname{\def\PY@tc##1{\textcolor[rgb]{0.00,0.63,0.00}{##1}}}
\expandafter\def\csname PY@tok@gr\endcsname{\def\PY@tc##1{\textcolor[rgb]{1.00,0.00,0.00}{##1}}}
\expandafter\def\csname PY@tok@ge\endcsname{\let\PY@it=\textit}
\expandafter\def\csname PY@tok@gs\endcsname{\let\PY@bf=\textbf}
\expandafter\def\csname PY@tok@gp\endcsname{\let\PY@bf=\textbf\def\PY@tc##1{\textcolor[rgb]{0.00,0.00,0.50}{##1}}}
\expandafter\def\csname PY@tok@go\endcsname{\def\PY@tc##1{\textcolor[rgb]{0.53,0.53,0.53}{##1}}}
\expandafter\def\csname PY@tok@gt\endcsname{\def\PY@tc##1{\textcolor[rgb]{0.00,0.27,0.87}{##1}}}
\expandafter\def\csname PY@tok@err\endcsname{\def\PY@bc##1{\setlength{\fboxsep}{0pt}\fcolorbox[rgb]{1.00,0.00,0.00}{1,1,1}{\strut ##1}}}
\expandafter\def\csname PY@tok@kc\endcsname{\let\PY@bf=\textbf\def\PY@tc##1{\textcolor[rgb]{0.00,0.50,0.00}{##1}}}
\expandafter\def\csname PY@tok@kd\endcsname{\let\PY@bf=\textbf\def\PY@tc##1{\textcolor[rgb]{0.00,0.50,0.00}{##1}}}
\expandafter\def\csname PY@tok@kn\endcsname{\let\PY@bf=\textbf\def\PY@tc##1{\textcolor[rgb]{0.00,0.50,0.00}{##1}}}
\expandafter\def\csname PY@tok@kr\endcsname{\let\PY@bf=\textbf\def\PY@tc##1{\textcolor[rgb]{0.00,0.50,0.00}{##1}}}
\expandafter\def\csname PY@tok@bp\endcsname{\def\PY@tc##1{\textcolor[rgb]{0.00,0.50,0.00}{##1}}}
\expandafter\def\csname PY@tok@fm\endcsname{\def\PY@tc##1{\textcolor[rgb]{0.00,0.00,1.00}{##1}}}
\expandafter\def\csname PY@tok@vc\endcsname{\def\PY@tc##1{\textcolor[rgb]{0.10,0.09,0.49}{##1}}}
\expandafter\def\csname PY@tok@vg\endcsname{\def\PY@tc##1{\textcolor[rgb]{0.10,0.09,0.49}{##1}}}
\expandafter\def\csname PY@tok@vi\endcsname{\def\PY@tc##1{\textcolor[rgb]{0.10,0.09,0.49}{##1}}}
\expandafter\def\csname PY@tok@vm\endcsname{\def\PY@tc##1{\textcolor[rgb]{0.10,0.09,0.49}{##1}}}
\expandafter\def\csname PY@tok@sa\endcsname{\def\PY@tc##1{\textcolor[rgb]{0.73,0.13,0.13}{##1}}}
\expandafter\def\csname PY@tok@sb\endcsname{\def\PY@tc##1{\textcolor[rgb]{0.73,0.13,0.13}{##1}}}
\expandafter\def\csname PY@tok@sc\endcsname{\def\PY@tc##1{\textcolor[rgb]{0.73,0.13,0.13}{##1}}}
\expandafter\def\csname PY@tok@dl\endcsname{\def\PY@tc##1{\textcolor[rgb]{0.73,0.13,0.13}{##1}}}
\expandafter\def\csname PY@tok@s2\endcsname{\def\PY@tc##1{\textcolor[rgb]{0.73,0.13,0.13}{##1}}}
\expandafter\def\csname PY@tok@sh\endcsname{\def\PY@tc##1{\textcolor[rgb]{0.73,0.13,0.13}{##1}}}
\expandafter\def\csname PY@tok@s1\endcsname{\def\PY@tc##1{\textcolor[rgb]{0.73,0.13,0.13}{##1}}}
\expandafter\def\csname PY@tok@mb\endcsname{\def\PY@tc##1{\textcolor[rgb]{0.40,0.40,0.40}{##1}}}
\expandafter\def\csname PY@tok@mf\endcsname{\def\PY@tc##1{\textcolor[rgb]{0.40,0.40,0.40}{##1}}}
\expandafter\def\csname PY@tok@mh\endcsname{\def\PY@tc##1{\textcolor[rgb]{0.40,0.40,0.40}{##1}}}
\expandafter\def\csname PY@tok@mi\endcsname{\def\PY@tc##1{\textcolor[rgb]{0.40,0.40,0.40}{##1}}}
\expandafter\def\csname PY@tok@il\endcsname{\def\PY@tc##1{\textcolor[rgb]{0.40,0.40,0.40}{##1}}}
\expandafter\def\csname PY@tok@mo\endcsname{\def\PY@tc##1{\textcolor[rgb]{0.40,0.40,0.40}{##1}}}
\expandafter\def\csname PY@tok@ch\endcsname{\let\PY@it=\textit\def\PY@tc##1{\textcolor[rgb]{0.25,0.50,0.50}{##1}}}
\expandafter\def\csname PY@tok@cm\endcsname{\let\PY@it=\textit\def\PY@tc##1{\textcolor[rgb]{0.25,0.50,0.50}{##1}}}
\expandafter\def\csname PY@tok@cpf\endcsname{\let\PY@it=\textit\def\PY@tc##1{\textcolor[rgb]{0.25,0.50,0.50}{##1}}}
\expandafter\def\csname PY@tok@c1\endcsname{\let\PY@it=\textit\def\PY@tc##1{\textcolor[rgb]{0.25,0.50,0.50}{##1}}}
\expandafter\def\csname PY@tok@cs\endcsname{\let\PY@it=\textit\def\PY@tc##1{\textcolor[rgb]{0.25,0.50,0.50}{##1}}}

\def\PYZbs{\char`\\}
\def\PYZus{\char`\_}
\def\PYZob{\char`\{}
\def\PYZcb{\char`\}}
\def\PYZca{\char`\^}
\def\PYZam{\char`\&}
\def\PYZlt{\char`\<}
\def\PYZgt{\char`\>}
\def\PYZsh{\char`\#}
\def\PYZpc{\char`\%}
\def\PYZdl{\char`\$}
\def\PYZhy{\char`\-}
\def\PYZsq{\char`\'}
\def\PYZdq{\char`\"}
\def\PYZti{\char`\~}
% for compatibility with earlier versions
\def\PYZat{@}
\def\PYZlb{[}
\def\PYZrb{]}
\makeatother


    % Exact colors from NB
    \definecolor{incolor}{rgb}{0.0, 0.0, 0.5}
    \definecolor{outcolor}{rgb}{0.545, 0.0, 0.0}



    
    % Prevent overflowing lines due to hard-to-break entities
    \sloppy 
    % Setup hyperref package
    \hypersetup{
      breaklinks=true,  % so long urls are correctly broken across lines
      colorlinks=true,
      urlcolor=urlcolor,
      linkcolor=linkcolor,
      citecolor=citecolor,
      }
    % Slightly bigger margins than the latex defaults
    
    \geometry{verbose,tmargin=1in,bmargin=1in,lmargin=1in,rmargin=1in}
    
    

    \begin{document}
    
    
    \maketitle
    
    

    
    \section{Mathematic Visualization}\label{mathematic-visualization}

\subsection{Table of contents}\label{table-of-contents}

\begin{enumerate}
\def\labelenumi{\arabic{enumi}.}
\tightlist
\item
  Section \ref{intro}
\item
  Section \ref{vizlandscape}
\item
  Section \ref{hv}
\item
  Section \ref{algo_viz}\\
  4.1 Section \ref{newton_method}\\
  4.2 Section \ref{fixed_point}\\
  4.3 Section \ref{lin_regression}\\
  4.4 Section \ref{num_integration}\\
\item
  Section \ref{matrix_viz}\\
  5.1 Section \ref{cmap_and_std_matrices}\\
  5.2 Section \ref{sym_matrices}\\
  5.3 Section \ref{gauss_elim}\\
  5.4 Section \ref{multi_dim_plotting}\\
\item
  Section \ref{summary}
\item
  Section \ref{bib}
\end{enumerate}

    \subsection{1. The Role of Mathematical Visualization
}\label{the-role-of-mathematical-visualization}

Mathematics is for many an abstract language. It is not uncommon to find
textbooks that are solely comprised of text and dense formulas. Often
forgotten is that the mathematical language can in most cases be
translated into or supported by visualizations in a variety of intuitive
ways with multiple benefits. Visualizations undoubtedly make the reading
experience more enjoyable and can significantly help and accelerate
understanding, increase curiosity and spark creativity. Approaching a
mathematical problem visually can also improve the research process by
encouraging a change of perspective. However, visualizations are
difficult to erase from one's memory, so if poorly constructed, they can
lead to confusion or if oversimplified, they can inhibit deeper
understanding.

Thus, a valuable visualization requires effort. Thanks to the ubiquity
of computers and rich open-source libraries, it has never been easier to
create beautiful looking and interactive visualizations. As a result,
adoption in recent years has increased in popularity as can for example
be attested by the success of the Youtube channel
\href{https://www.youtube.com/channel/UCYO_jab_esuFRV4b17AJtAw}{3Blue1Brown}.
Its author, Grant Sanderson, is able to break down and explain visually
complex mathematical concepts in brief videos that have attracted a wide
ranging audience. A further marvellous creation is the currently under
completion online textbook
\href{https://jermwatt.github.io/mlrefined/index.html}{"Machine Learning
Refined"} written by Jeremy Watt and Reza Borhani. It is richly
populated with interactive graphics and animations illustrating the
current state of the art machine learning methods.

    In a 2002 paper titled
\href{http://users.math.uoc.gr/~ictm2/Proceedings/invGuz.pdf}{"The Role
of Visualization: In the Teaching and Learning of Mathematical
Analysis"}, its author Miguel de Guzmán argues that visualization in
general goes beyond its concrete and easily visualizable aspect and
often involves many steps of interpretation and abstraction. For this,
Guzmán classifies visualization into four types that we briefly
introduce:

\textbf{\emph{Isomorphic visualization}}

An isomorphic visualization is achieved when an object has an "exact"
correspondence with its representation, i.e. the representation can be
translated directly into mathematical relationships. An example is the
visual proof of the Pythagoras Theorem.

    \begin{Verbatim}[commandchars=\\\{\}]
{\color{incolor}In [{\color{incolor}14}]:} \PY{k+kn}{from} \PY{n+nn}{py\PYZus{}code}\PY{n+nn}{.}\PY{n+nn}{pythagoras} \PY{k}{import} \PY{n}{pythagoras\PYZus{}viz}
         \PY{n}{pythagoras\PYZus{}viz}
\end{Verbatim}


\begin{Verbatim}[commandchars=\\\{\}]
{\color{outcolor}Out[{\color{outcolor}14}]:} :Layout
            .DynamicMap.I  :DynamicMap   [a]
               :Overlay
                  .Polygons.I   :Polygons   [x,y]
                  .Polygons.II  :Polygons   [x,y]
                  .Polygons.III :Polygons   [x,y]
                  .Labels.I     :Labels   [x,y]   (Label)
                  .Labels.II    :Labels   [x,y]   (Label)
            .DynamicMap.II :DynamicMap   [a]
               :Overlay
                  .Polygons.I  :Polygons   [x,y]
                  .Polygons.II :Polygons   [x,y]
                  .Labels.I    :Labels   [x,y]   (Label)
                  .Text.I      :Text   [x,y]
\end{Verbatim}
            
    Note the interactive widget on the right to enable variation of the
value \(a\) (with \(a+b:=100\)). Although the animation depicts special
cases, it is still clear, that the given relationships between \(a\),
\(b\) and \(c\) hold for any right triangle.

\textbf{\emph{Homeomorphic visualization}}

A homeomorphic visualization is when the representation of some elements
have certain mutual relations that imitate sufficiently well the
relationships between the abstract objects they describe.
\textless{}\textless{} example still missing, does a 2-D example to
explain an n-dimensional problem count? \textgreater{}\textgreater{}

\textbf{\emph{Analogical visualization}}

An analogical visualization is when we substitute the objects of
interest by others that relate between themselves in an analogous way,
but whose relationships are perhaps easier to understand. An example for
this type of visualization is contained in a
\href{https://www.youtube.com/watch?v=AmgkSdhK4K8}{possible proof} of
the \emph{inscribed rectangle problem}: to show is that there exist four
points on a closed loop that form a rectangle. This can be translated to
showing that there exist two pairs of points that share the same
mid-point - an equivalent definition of a rectangle.

\textbf{\emph{diagrammatic visualization}}

A diagrammatic visualization is when the objects and their mutual
relationships are merely represented by diagrams illustrating our
thinking process. An example is given in the beginning of the next
chapter depicting the python visualization landscape.

    We finish this introduction by stating the goal of this project and
laying down its structure. This project seeks to inspire the reader to
consider the usefulness of mathematical visualization and explore the
possibilities of current tools. For this, we make use of the programming
language Python and present currently existing libraries in
Section \ref{vizlandscape}. In Section \ref{hv}, we present our library
of choice, HoloViews, that we then use to visualize concepts from
introductory numeric courses. We focus on selected algorithms in
Section \ref{algo_viz} and explore matrix visualization in
Section \ref{matrix_viz}.

    \subsection{2. Python Visualization Landscape
}\label{python-visualization-landscape}

In this section, we explore the visualization libraries available in
Python. Here, one is easily overwelmed by the sea of possibilities as
can be witnessed in the following map:

    

    It was presented at the Python conference Pycon 2017 (with a recorded
presentation available on
\href{https://www.youtube.com/watch?v=TPl9bMg8j8U}{Youtube}). As of
today, there is no tool that serves all visualization purposes. Each
tool has its advantages and disadvantages and a compromise has to be
made depending on ones visualization goal. We briefly discuss some of
the available options.

\paragraph{\texorpdfstring{\emph{Matplotlib}}{Matplotlib}}\label{matplotlib}

Matplotlib is one of the oldest visualization tools in Python and was
intended to resemble and possibly substitute Matlab's visualization
tool. Over the years, several options of tools have been built on top
for specialized application, for example \textbf{\emph{Seaborn}} for
statistical visualizations and \textbf{\emph{NetworkX}} for network
visualizations. While highly robust and versatile, Matplotlib is also
quite verbose. Most of the tools that have spawned since are to
alleviate some of its deficiencies.

\paragraph{\texorpdfstring{\emph{JavaScript}}{JavaScript}}\label{javascript}

Since widely used on the Internet, JavaScript is optimized for browser
use and can be leveraged to render graphics for online publication,
making it particularly attractive for Jupyter Notebook applications.
Here, the libraries \textbf{\emph{PythreeJS}} and
\textbf{\emph{IPyVolume}} allow for efficient 3D rendering and
\textbf{\emph{IPyleaflet}} enables the import of maps into the Notebook.
The libraries \textbf{\emph{Bokeh}}, \textbf{\emph{Plotly}} and
\textbf{\emph{bqplot}} all offer high quality interactive data
visualization tools. Plotly is similar to Bokeh with additionally good
3D plotting capabilities, but requires paid plans for some features. The
library \textbf{\emph{Cufflinks}}, built on top of Plotly, optimizes
interactive charting with \textbf{\emph{Pandas}} and is often used for
visualizing financial data, as is bqplot (which is being developed by
the financial media company Bloomberg).

\paragraph{\texorpdfstring{\emph{Datashader} and
\emph{Vaex}}{Datashader and Vaex}}\label{datashader-and-vaex}

Datashader and Vaex enable efficient interactive visualization of large
data sets.

\paragraph{\texorpdfstring{\emph{Vega} and
\emph{Altair}}{Vega and Altair}}\label{vega-and-altair}

Vega is a visualization grammar. Rather than forcing the user to specify
\emph{how} a visualization should be constructed, its intend is to let
the user focus on \emph{what} should be done. Data and relationships are
put in the foreground and leave out tinkering with incidental details -
at least in the first steps of data exploration. Altair is built on top
of Vega and enables visualizations with minimal amount of code in an
intuitive way. Additionally, instead of saving visualizations in the
form of pixels, the raw data is kept in the background, enabling
interaction with the data and facilitating eventual formatting changes.

\paragraph{\texorpdfstring{\emph{HoloViews}}{HoloViews}}\label{holoviews}

Similar to Altair, HoloViews also enables interaction with data on a
higher level. Once a type of plot is specified for a particular data
set, HoloViews intelligently visualizes the data with no further user
input required. Additionally, interactive visualizations are also easily
made possible with just a few lines of code. HoloViews can use Bokeh,
Plotly and Matplotlib as back-end.

\paragraph{\texorpdfstring{\emph{ggpy}}{ggpy}}\label{ggpy}

The tool ggpy seeks to also offer a higher level of user interaction for
statistical visualization.

\paragraph{\texorpdfstring{\emph{OpenGL} and
\emph{Vispy}}{OpenGL and Vispy}}\label{opengl-and-vispy}

OpenGL is a cross-platform application programming interface (API) for
rendering 2D and 3D vector graphics which Vispy takes advantage of.
OpenGL is typically graphics processing unit (GPU) accelerated, enabling
to visualize graphic intensive applications. It is for example often
used for computer-aided design (CAD), virtual reality and video games.

\paragraph{Other Visualization Tools}\label{other-visualization-tools}

The tools in the lower left part in the above map have not been further
categorized. They represent to some extend competing libraries to those
described and often differ in specific characteristics like speed or
cross-platform compatibility. \textbf{\emph{Mayavi}} is for example
similar to Vispy and enables advanced 3D visualization capabilities.
Some tools are being developed in collaboration with others like
\textbf{\emph{PyQtGraph}} and Vispy. Note that the links between
libraries are not limited to the ones visualized on the map, they often
interconnect in myriads of ways.

    As with all ecosystems, the visualization landscape constantly evolves.
Some libraries will merge features and some will stop being developed.
The heat map and function plot below show the GitHub commit activity of
all the libraries presented in the previous map (except OpenGL,
JavaScript, pandas and graphviz) over the past 52 weeks (the data was
gathered in August 2018). The rows of the heat map correspond to the
individual libraries and have been sorted by their corresponding total
number of commits over the past year. The columns represent the weeks
and the colors show the amount of commits per week per library (yellow
for a high number, black for no commit). The function plot underneath
the heat map charts the weekly commits for a specific library. This
library can be changed by clicking through the rows of the heat map. 

    \begin{Verbatim}[commandchars=\\\{\}]
{\color{incolor}In [{\color{incolor}15}]:} \PY{k+kn}{from} \PY{n+nn}{py\PYZus{}code}\PY{n+nn}{.}\PY{n+nn}{repo\PYZus{}activity} \PY{k}{import} \PY{n}{repo\PYZus{}activity\PYZus{}viz}  \PY{c+c1}{\PYZsh{} import code for creating visualization}
         \PY{n}{repo\PYZus{}activity\PYZus{}viz}
\end{Verbatim}


\begin{Verbatim}[commandchars=\\\{\}]
{\color{outcolor}Out[{\color{outcolor}15}]:} :Layout
            .HeatMap.Repository\_Activity :HeatMap   [week,library]   (commits)
            .DynamicMap.I                :DynamicMap   []
               :Curve   [week]   (commits)
\end{Verbatim}
            
    From the heat map we observe that libraries such as ggpy and
\textbf{\emph{d3py}} are no longer actively developed while Matplotlib
remains indisputably the library with the most commits. (Note that we
left out \textbf{\emph{pandas}} from the above visualization, as the
library focuses mainly on non-visulization tools and its high number of
commits would skew the visualization.)

    \section{3. Introduction to HoloViews }\label{introduction-to-holoviews}

As described in the previous section, thanks to a high level syntax,
HoloViews enables to explore data and manipulate objects in a relatively
intuitive way with few lines of code. Additionally, it is actively being
developed to meet the many needs of data scientists and developers for
use in both the Jupyter Notebook and outside. Furthermore, it is tightly
integrated and co-developed with other libraries to allow working with
large data sets seamlessly. The
\href{http://pyviz.org/background.html}{PyViz Project} illustrates the
possible workflows within the HoloViews surrounding ecosystem as
following:

    For the remainder of this project, we use the HoloViews library and use
Bokeh as its back-end in the Jupyter Notebook. To familiarize oneself
with the library, we recommend the section
\href{https://gitlab.gwdg.de/jschulz1/math_prog/blob/master/lecture/22_holoviews.ipynb}{Mathematisch
orientiertes Programmieren - Holoviews} by Jochen Schulz's lecture (in
German). An
\href{http://holoviews.org/getting_started/index.html}{introduction} is
also provided on the \href{http://holoviews.org/}{HoloViews website},
with many more resources, in particular beautiful galleries
\href{http://holoviews.org/gallery/index.html}{here} and
\href{http://holoviews.org/reference/index.html}{here} (or Bokeh's
gallery
\href{https://bokeh.pydata.org/en/latest/docs/gallery.html}{here}).
Lastly, we highly recommend any reader seriously considering HoloViews
to go through the comprehensive
\href{http://pyviz.org/tutorial/index.html}{PyViz Tutorial} to learn
more about the finer controls of the tool, as in particular the plot
styling can lead to difficulties in the beginning. Held by one of the
core developers, the tutorial is also available in
\href{https://www.youtube.com/watch?time_continue=33\&v=aZ1G_Q7ovmc}{video
format}.

    From here on, we assume the reader to be accustomed to the HoloViews
syntax on a basic level. We adopt the standard import abbreviations for
\texttt{holoviews} and \href{http://www.numpy.org/}{\texttt{numpy}} as
follows

    \begin{Verbatim}[commandchars=\\\{\}]
{\color{incolor}In [{\color{incolor}16}]:} \PY{k+kn}{import} \PY{n+nn}{holoviews} \PY{k}{as} \PY{n+nn}{hv}
         \PY{n}{hv}\PY{o}{.}\PY{n}{extension}\PY{p}{(}\PY{l+s+s1}{\PYZsq{}}\PY{l+s+s1}{bokeh}\PY{l+s+s1}{\PYZsq{}}\PY{p}{)}  \PY{c+c1}{\PYZsh{} HoloViews\PYZsq{} back\PYZhy{}end}
         \PY{k+kn}{import} \PY{n+nn}{numpy} \PY{k}{as} \PY{n+nn}{np}
\end{Verbatim}


    
    
    
    
    
    
    Furthermore, during the course of this project, we also use the
libraries \href{https://pandas.pydata.org/}{\texttt{pandas}},
\href{https://github.com/HIPS/autograd}{\texttt{autograd}},
\href{https://www.scipy.org/}{\texttt{scipy}}
\href{https://docs.python.org/2/library/copy.html}{\texttt{copy}} and
\href{https://github.com/pyviz/colorcet}{\texttt{colorcet}}.

For our first example, we display in detail our code and the used
options. Following on, we hide the code in separate Python files (saved
in the folder "py\_code") and import them as seen previously. For
example as follows:

\begin{Shaded}
\begin{Highlighting}[]
\ImportTok{from}\NormalTok{ py_code.example }\ImportTok{import}\NormalTok{ example}
\end{Highlighting}
\end{Shaded}

    \section{4. Algorithm Visualization }\label{algorithm-visualization}

\subsection{4.1. Newton's Method }\label{newtons-method}

Newton's method is a method for finding successively better
approximations to the roots of a real-valued function, say \(f\). I.e.
the method seeks to find \(x\) such that \(f(x) = 0\) or close to. Here,
we assume \(f:[a,b] \rightarrow \mathbb{R}\) to be a differentiable
function defined on the interval \([a,b]\).

We begin by defining a function \(f\) (with real root) we seek to
visualize and plot it below. We choose
\(f(x)= \frac{1}{10}x^2 - x + 1\).

    \begin{Verbatim}[commandchars=\\\{\}]
{\color{incolor}In [{\color{incolor}17}]:} \PY{c+c1}{\PYZsh{} define function f}
         \PY{k}{def} \PY{n+nf}{f}\PY{p}{(}\PY{n}{x}\PY{p}{)}\PY{p}{:}
             \PY{k}{return} \PY{n}{x}\PY{o}{*}\PY{o}{*}\PY{l+m+mi}{2}\PY{o}{/}\PY{l+m+mf}{10.0} \PY{o}{\PYZhy{}} \PY{n}{x} \PY{o}{+} \PY{l+m+mi}{1}
         
         \PY{c+c1}{\PYZsh{} create points}
         \PY{n}{min\PYZus{}x} \PY{o}{=} \PY{o}{\PYZhy{}}\PY{l+m+mf}{10.0}
         \PY{n}{max\PYZus{}x} \PY{o}{=} \PY{l+m+mf}{12.0}
         \PY{n}{x} \PY{o}{=} \PY{n}{np}\PY{o}{.}\PY{n}{linspace}\PY{p}{(}\PY{n}{min\PYZus{}x}\PY{p}{,} \PY{n}{max\PYZus{}x}\PY{p}{,} \PY{n}{num} \PY{o}{=} \PY{l+m+mi}{200}\PY{p}{)}
         \PY{n}{y} \PY{o}{=} \PY{n}{f}\PY{p}{(}\PY{n}{x}\PY{p}{)}
         
         \PY{c+c1}{\PYZsh{} create HoloViews objects}
         \PY{n}{hv\PYZus{}f} \PY{o}{=} \PY{n}{hv}\PY{o}{.}\PY{n}{Curve}\PY{p}{(}\PY{p}{(}\PY{n}{x}\PY{p}{,} \PY{n}{y}\PY{p}{)}\PY{p}{)}
         \PY{n}{hv\PYZus{}hline} \PY{o}{=} \PY{n}{hv}\PY{o}{.}\PY{n}{HLine}\PY{p}{(}\PY{l+m+mi}{0}\PY{p}{)}
         
         \PY{c+c1}{\PYZsh{} generate plot}
         \PY{n}{hv\PYZus{}f} \PY{o}{*} \PY{n}{hv\PYZus{}hline}
\end{Verbatim}


\begin{Verbatim}[commandchars=\\\{\}]
{\color{outcolor}Out[{\color{outcolor}17}]:} :Overlay
            .Curve.I :Curve   [x]   (y)
            .HLine.I :HLine   [x,y]
\end{Verbatim}
            
    This output is based on the default plotting parameters. Improving the
visual appeal of this plot, unfortunately, becomes more tedious, but
could be done in the following way:

    \begin{Verbatim}[commandchars=\\\{\}]
{\color{incolor}In [{\color{incolor}18}]:} \PY{c+c1}{\PYZsh{} optimize plotting ranges to accomodate for coming tangent}
         \PY{n}{range\PYZus{}x} \PY{o}{=} \PY{n}{max\PYZus{}x} \PY{o}{\PYZhy{}} \PY{n}{min\PYZus{}x}
         \PY{n}{range\PYZus{}expand\PYZus{}x} \PY{o}{=} \PY{l+m+mf}{0.02}  \PY{c+c1}{\PYZsh{} increase x\PYZhy{}range by 2\PYZpc{} }
         \PY{n}{min\PYZus{}plot\PYZus{}x} \PY{o}{=} \PY{n}{min\PYZus{}x} \PY{o}{\PYZhy{}} \PY{n}{range\PYZus{}x} \PY{o}{*} \PY{n}{range\PYZus{}expand\PYZus{}x}
         \PY{n}{max\PYZus{}plot\PYZus{}x} \PY{o}{=} \PY{n}{max\PYZus{}x} \PY{o}{+} \PY{n}{range\PYZus{}x} \PY{o}{*} \PY{n}{range\PYZus{}expand\PYZus{}x}
         
         \PY{n}{min\PYZus{}y} \PY{o}{=} \PY{n}{np}\PY{o}{.}\PY{n}{min}\PY{p}{(}\PY{n}{y}\PY{p}{)}
         \PY{n}{max\PYZus{}y} \PY{o}{=} \PY{n}{np}\PY{o}{.}\PY{n}{max}\PY{p}{(}\PY{n}{y}\PY{p}{)}
         \PY{n}{range\PYZus{}y} \PY{o}{=} \PY{n}{max\PYZus{}y} \PY{o}{\PYZhy{}} \PY{n}{min\PYZus{}y}
         \PY{n}{range\PYZus{}expand\PYZus{}y}  \PY{o}{=} \PY{l+m+mf}{0.1}  \PY{c+c1}{\PYZsh{} increase y\PYZhy{}range by 10\PYZpc{}}
         \PY{n}{min\PYZus{}plot\PYZus{}y} \PY{o}{=} \PY{n}{min\PYZus{}y} \PY{o}{\PYZhy{}} \PY{n}{range\PYZus{}y} \PY{o}{*} \PY{n}{range\PYZus{}expand\PYZus{}y}
         \PY{n}{max\PYZus{}plot\PYZus{}y} \PY{o}{=} \PY{n}{max\PYZus{}y} \PY{o}{+} \PY{n}{range\PYZus{}y} \PY{o}{*} \PY{n}{range\PYZus{}expand\PYZus{}y}
         
         \PY{n}{hv\PYZus{}f} \PY{o}{=} \PY{n}{hv\PYZus{}f}\PY{o}{.}\PY{n}{redim}\PY{o}{.}\PY{n}{range}\PY{p}{(}\PY{n}{x}\PY{o}{=}\PY{p}{(}\PY{n}{min\PYZus{}plot\PYZus{}x}\PY{p}{,} \PY{n}{max\PYZus{}plot\PYZus{}x}\PY{p}{)}\PY{p}{,} \PY{n}{y}\PY{o}{=}\PY{p}{(}\PY{n}{min\PYZus{}plot\PYZus{}y}\PY{p}{,} \PY{n}{max\PYZus{}plot\PYZus{}y}\PY{p}{)}\PY{p}{)}  \PY{c+c1}{\PYZsh{} apply changes to figure}
         
         \PY{c+c1}{\PYZsh{} rename axis}
         \PY{n}{hv\PYZus{}f} \PY{o}{=} \PY{n}{hv\PYZus{}f}\PY{o}{.}\PY{n}{redim}\PY{p}{(}\PY{n}{y}\PY{o}{=}\PY{l+s+s1}{\PYZsq{}}\PY{l+s+s1}{f(x)}\PY{l+s+s1}{\PYZsq{}}\PY{p}{)}
         
         \PY{c+c1}{\PYZsh{} fine tune plot styling}
         \PY{k+kn}{import} \PY{n+nn}{bokeh}\PY{n+nn}{.}\PY{n+nn}{palettes} \PY{k}{as} \PY{n+nn}{bp}  \PY{c+c1}{\PYZsh{} import color palettes for beautiful colors}
         \PY{n}{color\PYZus{}palette} \PY{o}{=} \PY{l+s+s1}{\PYZsq{}}\PY{l+s+s1}{Inferno}\PY{l+s+s1}{\PYZsq{}}
         \PY{n}{color\PYZus{}1} \PY{o}{=} \PY{n}{bp}\PY{o}{.}\PY{n}{all\PYZus{}palettes}\PY{p}{[}\PY{n}{color\PYZus{}palette}\PY{p}{]}\PY{p}{[}\PY{l+m+mi}{256}\PY{p}{]}\PY{p}{[}\PY{l+m+mi}{145}\PY{p}{]}
         \PY{n}{color\PYZus{}2} \PY{o}{=} \PY{n}{bp}\PY{o}{.}\PY{n}{all\PYZus{}palettes}\PY{p}{[}\PY{n}{color\PYZus{}palette}\PY{p}{]}\PY{p}{[}\PY{l+m+mi}{256}\PY{p}{]}\PY{p}{[}\PY{l+m+mi}{20}\PY{p}{]}
         
         \PY{c+c1}{\PYZsh{} plot styling options}
         \PY{n}{styling\PYZus{}options} \PY{o}{=} \PY{p}{\PYZob{}}\PY{l+s+s1}{\PYZsq{}}\PY{l+s+s1}{Curve}\PY{l+s+s1}{\PYZsq{}}\PY{p}{:} \PY{n+nb}{dict}\PY{p}{(}\PY{n}{height}\PY{o}{=}\PY{l+m+mi}{400}\PY{p}{,} \PY{n}{width}\PY{o}{=}\PY{l+m+mi}{640}\PY{p}{,} \PY{n}{tools}\PY{o}{=}\PY{p}{[}\PY{l+s+s1}{\PYZsq{}}\PY{l+s+s1}{hover}\PY{l+s+s1}{\PYZsq{}}\PY{p}{]}\PY{p}{,} \PY{n}{toolbar}\PY{o}{=}\PY{l+s+s1}{\PYZsq{}}\PY{l+s+s1}{right}\PY{l+s+s1}{\PYZsq{}}\PY{p}{,} \PY{n}{color}\PY{o}{=}\PY{n}{color\PYZus{}1}\PY{p}{,} \PY{n}{line\PYZus{}width}\PY{o}{=}\PY{l+m+mf}{2.5}\PY{p}{,}\PYZbs{}
                                         \PY{n}{title\PYZus{}format}\PY{o}{=}\PY{l+s+s1}{\PYZsq{}}\PY{l+s+s1}{Function Plot}\PY{l+s+s1}{\PYZsq{}}\PY{p}{)}\PY{p}{,}
                            \PY{l+s+s1}{\PYZsq{}}\PY{l+s+s1}{HLine}\PY{l+s+s1}{\PYZsq{}}\PY{p}{:} \PY{n+nb}{dict}\PY{p}{(}\PY{n}{color}\PY{o}{=}\PY{l+s+s1}{\PYZsq{}}\PY{l+s+s1}{grey}\PY{l+s+s1}{\PYZsq{}}\PY{p}{,} \PY{n}{line\PYZus{}width}\PY{o}{=}\PY{l+m+mi}{2}\PY{p}{)}\PY{p}{\PYZcb{}}
         
         \PY{c+c1}{\PYZsh{} generate customized plot}
         \PY{n}{layout\PYZus{}f} \PY{o}{=} \PY{n}{hv\PYZus{}f} \PY{o}{*} \PY{n}{hv\PYZus{}hline}
         \PY{n}{layout\PYZus{}f} \PY{o}{=} \PY{n}{layout\PYZus{}f}\PY{o}{.}\PY{n}{options}\PY{p}{(}\PY{n}{styling\PYZus{}options}\PY{p}{)}  \PY{c+c1}{\PYZsh{} add styling options}
         \PY{n}{layout\PYZus{}f}
\end{Verbatim}


\begin{Verbatim}[commandchars=\\\{\}]
{\color{outcolor}Out[{\color{outcolor}18}]:} :Overlay
            .Curve.I :Curve   [x]   (f(x))
            .HLine.I :HLine   [x,y]
\end{Verbatim}
            
    Here, we activated the hover-tool. By dragging the cursor over the
function, we can already roughly estimate its two roots (\(1.055\) and
\(8.905\)). It is also possible to use the zoom-tool to get an even
better estimates.

Since Newton's method requires the derivative of \(f\), we denote by
\(f'\), we use it to compute the tangent of \(f(x)\) for every point
\(x\) and plot it as an interactive graphic.

    \begin{Verbatim}[commandchars=\\\{\}]
{\color{incolor}In [{\color{incolor}19}]:} \PY{k+kn}{from} \PY{n+nn}{autograd} \PY{k}{import} \PY{n}{grad}
         
         \PY{c+c1}{\PYZsh{} get derivative}
         \PY{n}{f\PYZus{}prime} \PY{o}{=} \PY{n}{grad}\PY{p}{(}\PY{n}{f}\PY{p}{)}
         
         \PY{c+c1}{\PYZsh{} define the tangent to f at x\PYZus{}1}
         \PY{k}{def} \PY{n+nf}{tangent\PYZus{}f}\PY{p}{(}\PY{n}{x}\PY{p}{,} \PY{n}{x\PYZus{}1}\PY{p}{)}\PY{p}{:}
             \PY{k}{return} \PY{n}{f\PYZus{}prime}\PY{p}{(}\PY{n}{x\PYZus{}1}\PY{p}{)}\PY{o}{*}\PY{p}{(}\PY{n}{x}\PY{o}{\PYZhy{}}\PY{n}{x\PYZus{}1}\PY{p}{)} \PY{o}{+} \PY{n}{f}\PY{p}{(}\PY{n}{x\PYZus{}1}\PY{p}{)}
         
         \PY{c+c1}{\PYZsh{} create HoloViews object}
         \PY{k}{def} \PY{n+nf}{hv\PYZus{}tangent\PYZus{}f}\PY{p}{(}\PY{n}{x\PYZus{}1}\PY{p}{)}\PY{p}{:}
             \PY{k}{return} \PY{n}{hv}\PY{o}{.}\PY{n}{Curve}\PY{p}{(}\PY{p}{(}\PY{n}{x}\PY{p}{,} \PY{n}{tangent\PYZus{}f}\PY{p}{(}\PY{n}{x}\PY{p}{,} \PY{n}{x\PYZus{}1}\PY{p}{)}\PY{p}{)}\PY{p}{)}\PY{o}{.}\PY{n}{options}\PY{p}{(}\PY{n}{color} \PY{o}{=} \PY{n}{color\PYZus{}2}\PY{p}{)}
         \PY{c+c1}{\PYZsh{} create vertical line}
         \PY{k}{def} \PY{n+nf}{hv\PYZus{}vline}\PY{p}{(}\PY{n}{x\PYZus{}1}\PY{p}{)}\PY{p}{:}
             \PY{k}{return} \PY{n}{hv}\PY{o}{.}\PY{n}{VLine}\PY{p}{(}\PY{n}{x\PYZus{}1}\PY{p}{)}
         
         \PY{c+c1}{\PYZsh{} create DynamicMap for tangent}
         \PY{n}{default\PYZus{}x\PYZus{}1} \PY{o}{=} \PY{n}{min\PYZus{}x} \PY{o}{+} \PY{n}{range\PYZus{}x}\PY{o}{*}\PY{l+m+mf}{0.2}  \PY{c+c1}{\PYZsh{} default plotting value dynamic map}
         \PY{n}{dmap} \PY{o}{=} \PY{n}{hv}\PY{o}{.}\PY{n}{DynamicMap}\PY{p}{(}\PY{n}{hv\PYZus{}tangent\PYZus{}f}\PY{p}{,} \PY{n}{kdims}\PY{o}{=}\PY{p}{[}\PY{n}{hv}\PY{o}{.}\PY{n}{Dimension}\PY{p}{(}\PY{l+s+s1}{\PYZsq{}}\PY{l+s+s1}{x\PYZus{}1}\PY{l+s+s1}{\PYZsq{}}\PY{p}{,} \PY{n+nb}{range}\PY{o}{=}\PY{p}{(}\PY{n}{min\PYZus{}x}\PY{p}{,} \PY{n}{max\PYZus{}x}\PY{p}{)}\PY{p}{,} \PY{n}{default}\PY{o}{=}\PY{n}{default\PYZus{}x\PYZus{}1}\PY{p}{)}\PY{p}{]}\PY{p}{)}
         
         \PY{c+c1}{\PYZsh{} create DynamicMap for vertical line}
         \PY{n}{dmap\PYZus{}vline} \PY{o}{=} \PY{n}{hv}\PY{o}{.}\PY{n}{DynamicMap}\PY{p}{(}\PY{n}{hv\PYZus{}vline}\PY{p}{,} \PY{n}{kdims} \PY{o}{=} \PY{p}{[}\PY{n}{hv}\PY{o}{.}\PY{n}{Dimension}\PY{p}{(}\PY{l+s+s1}{\PYZsq{}}\PY{l+s+s1}{x\PYZus{}1}\PY{l+s+s1}{\PYZsq{}}\PY{p}{,} \PY{n+nb}{range}\PY{o}{=}\PY{p}{(}\PY{n}{min\PYZus{}x}\PY{p}{,} \PY{n}{max\PYZus{}x}\PY{p}{)}\PY{p}{,} \PY{n}{default}\PY{o}{=}\PY{n}{default\PYZus{}x\PYZus{}1}\PY{p}{)}\PY{p}{]}\PY{p}{)}
         \PY{n}{opts\PYZus{}vline} \PY{o}{=} \PY{p}{\PYZob{}}\PY{l+s+s1}{\PYZsq{}}\PY{l+s+s1}{VLine}\PY{l+s+s1}{\PYZsq{}}\PY{p}{:} \PY{n+nb}{dict}\PY{p}{(}\PY{n}{color}\PY{o}{=}\PY{l+s+s1}{\PYZsq{}}\PY{l+s+s1}{grey}\PY{l+s+s1}{\PYZsq{}}\PY{p}{,} \PY{n}{line\PYZus{}width}\PY{o}{=}\PY{l+m+mf}{0.5}\PY{p}{,} \PY{n}{line\PYZus{}dash}\PY{o}{=}\PY{l+s+s1}{\PYZsq{}}\PY{l+s+s1}{dashed}\PY{l+s+s1}{\PYZsq{}}\PY{p}{)}\PY{p}{\PYZcb{}}
         \PY{n}{dmap\PYZus{}vline} \PY{o}{=} \PY{n}{dmap\PYZus{}vline}\PY{o}{.}\PY{n}{options}\PY{p}{(}\PY{n}{opts\PYZus{}vline}\PY{p}{)}  \PY{c+c1}{\PYZsh{} add styling options}
         
         \PY{c+c1}{\PYZsh{} generate final function plot + tangent}
         \PY{n}{layout\PYZus{}tangent} \PY{o}{=} \PY{n}{layout\PYZus{}f} \PY{o}{*} \PY{n}{dmap} \PY{o}{*} \PY{n}{dmap\PYZus{}vline}
         \PY{n}{layout\PYZus{}tangent} \PY{o}{=} \PY{n}{layout\PYZus{}tangent}\PY{o}{.}\PY{n}{redim}\PY{o}{.}\PY{n}{range}\PY{p}{(}\PY{n}{x} \PY{o}{=} \PY{p}{(}\PY{n}{min\PYZus{}plot\PYZus{}x}\PY{p}{,} \PY{n}{max\PYZus{}plot\PYZus{}x}\PY{p}{)}\PY{p}{,} \PY{n}{y} \PY{o}{=} \PY{p}{(}\PY{n}{min\PYZus{}plot\PYZus{}y}\PY{p}{,} \PY{n}{max\PYZus{}plot\PYZus{}y}\PY{p}{)}\PY{p}{)}
         \PY{n}{layout\PYZus{}tangent}\PY{o}{.}\PY{n}{options}\PY{p}{(}\PY{n}{title\PYZus{}format}\PY{o}{=}\PY{l+s+s1}{\PYZsq{}}\PY{l+s+s1}{Tangent Plot}\PY{l+s+s1}{\PYZsq{}}\PY{p}{)}
\end{Verbatim}


\begin{Verbatim}[commandchars=\\\{\}]
{\color{outcolor}Out[{\color{outcolor}19}]:} :DynamicMap   [x\_1]
            :Overlay
               .Curve.I  :Curve   [x]   (f(x))
               .HLine.I  :HLine   [x,y]
               .Curve.II :Curve   [x]   (y)
               .VLine.I  :VLine   [x,y]
\end{Verbatim}
            
    With the help of the widget on the right, we can vary the value \(x\)
for which to draw the tangent of \(f(x)\).

    The idea of Newton's method consists in beginning with a reasonable
initial guess for \(x\), say \(x_0\), that is not too distant from the
root we seek to find. We then approximate the function by its tangent at
the point \(x_0\). With elementary algebra, we then determine the
\(x\)-intercept, say \(x_1\), of the tangent. The value \(x_1\) is then
subsequently used and the process is repeated. Now, suppose that we have
some current approximation \(x_n\). We obtain the next better
approximation \(x_{n+1}\) with the following formula

\[x_{n+1} = x_n - \frac{f(x_n)}{f'(x_n)}.\]

Implemented for \(x_0 = -8.5\) and \(6\) steps, we obtain the following
visualization of the algorithm.

    \begin{Verbatim}[commandchars=\\\{\}]
{\color{incolor}In [{\color{incolor}20}]:} \PY{k+kn}{from} \PY{n+nn}{py\PYZus{}code}\PY{n+nn}{.}\PY{n+nn}{newton\PYZus{}algo} \PY{k}{import} \PY{n}{newton\PYZus{}viz}
         \PY{n}{newton\PYZus{}viz}
\end{Verbatim}


    
    
    
    
    
    
\begin{Verbatim}[commandchars=\\\{\}]
{\color{outcolor}Out[{\color{outcolor}20}]:} :HoloMap   [Steps]
            :Overlay
               .Curve.I    :Curve   [x]   (y)
               .VLine.I    :VLine   [x,y]
               .VLine.II   :VLine   [x,y]
               .Points.I   :Points   [x,y]
               .Points.II  :Points   [x,y]
               .Points.III :Points   [x,y]
               .Curve.II   :Curve   [x]   (f(x))
               .HLine.I    :HLine   [x,y]
\end{Verbatim}
            
    For this example, we see that the algorithm converges very rapidly as it
is already close to the root by step \(3\).

    \subsection{4.2 Fixed-point Iteration }\label{fixed-point-iteration}

    Assume a function \(f\). Then, the fixed-point iteration is a method for
finding a point \(x\), if it exists, such that \(f(x)=x\) by using an
iteration process. Given a starting point \(x_0\), the fixed-point
iteration is given by

\[x_{n+1} = f(x_n), \text{ } n=0,1,2,\dots.\]

The resulting sequence \(x_0, x_1, \dots\) is hoped to converge to the
fixed-point \(x\). By the \emph{Banach fixed-point theorem}, this is the
case if the function \(f\) satisfies the following condition

\[\left|f(x_1)-f(x_2)\right|< L\left|x_1-x_2\right|,\]

where \(L\) is the \emph{Lipschitz constant} and we require \(L<1\).

Now, let us consider the special case, where \(f(x):= ax+b\). Then, the
above condition simplifies to

\[ \left|f(x_1)-f(x_2)\right|< \left|x_1-x_2\right|\]

\[ \iff \frac{\left|f(x_1)-f(x_2)\right|}{\left|x_1-x_2\right|} < 1 \]

\[ \iff a < 1.\]

Below, we visualize this condition for the fixed-point iteration by
setting \(x_0=8, b=2\) and letting \(a\) vary between \(0.4\) and
\(1.2\) (\(a=0\) in on the widget corresponds to \(a=0.4\) in the
drawing and \(a=20\) corresponds to \(a=1.2\)):

    \begin{Verbatim}[commandchars=\\\{\}]
{\color{incolor}In [{\color{incolor}4}]:} \PY{k+kn}{from} \PY{n+nn}{py\PYZus{}code}\PY{n+nn}{.}\PY{n+nn}{fixed\PYZus{}point\PYZus{}iteration} \PY{k}{import} \PY{n}{fix\PYZus{}point\PYZus{}viz\PYZus{}1}\PY{p}{,} \PY{n}{fix\PYZus{}point\PYZus{}viz\PYZus{}2}
        \PY{n}{fix\PYZus{}point\PYZus{}viz\PYZus{}1}
\end{Verbatim}


\begin{Verbatim}[commandchars=\\\{\}]
{\color{outcolor}Out[{\color{outcolor}4}]:} :HoloMap   [a,step]
           :Overlay
              .Curve.I  :Curve   [x]   (y)
              .Curve.II :Curve   [x]   (y)
              .Path.I   :Path   [x,y]
\end{Verbatim}
            
    \textless{}\textless{} Some plot optimizations and explanations on the
way \textgreater{}\textgreater{}

    \begin{Verbatim}[commandchars=\\\{\}]
{\color{incolor}In [{\color{incolor}5}]:} \PY{n}{fix\PYZus{}point\PYZus{}viz\PYZus{}2}
\end{Verbatim}


\begin{Verbatim}[commandchars=\\\{\}]
{\color{outcolor}Out[{\color{outcolor}5}]:} :HoloMap   [a,step]
           :Overlay
              .Curve.I  :Curve   [x]   (y)
              .Curve.II :Curve   [x]   (y)
              .Path.I   :Path   [x,y]
\end{Verbatim}
            
    From the animation, we can confirm that the fixed-point iteration series
converges for \(a<1\) and diverges for \(a>1\).

Note that the Newton Algorithm is a special case of the fixed-point
iteration, by using the function \(g(x):= x - \frac{f(x)}{f\prime(x)}\)
in the iteration process.

    \subsection{4.3. Least Squares Linear Regression
}\label{least-squares-linear-regression}

    In the two-dimensional case, a linear regression consists in the fitting
of a representative line to a data set of the form
\((x_1,y_1),(x_2,y_2),\dots, (x_n,y_n), n \in \mathbb{N}\), where
\((x_i,y_i), i=1,\dots,n\) denotes the \(i\)-th data point in the data
set. Expressed mathematically, we seek to find a slope
\(a \in \mathbb{R}\) and an offset \(b \in \mathbb{R}\) such that the
following relationship holds
\[ax_i+b \approx y_i, \text{     for }i=1,\dots,n.\]

Or visually, for a data set consisting of \(n=10\) points:

    \begin{Verbatim}[commandchars=\\\{\}]
{\color{incolor}In [{\color{incolor}21}]:} \PY{k+kn}{from} \PY{n+nn}{py\PYZus{}code}\PY{n+nn}{.}\PY{n+nn}{least\PYZus{}squares} \PY{k}{import} \PY{n}{hv\PYZus{}points\PYZus{}viz}\PY{p}{,} \PY{n}{least\PYZus{}square\PYZus{}viz}
         \PY{n}{hv\PYZus{}points\PYZus{}viz}
\end{Verbatim}


\begin{Verbatim}[commandchars=\\\{\}]
{\color{outcolor}Out[{\color{outcolor}21}]:} :Overlay
            .Points.I                                  :Points   [x,y]
            .Curve.Sum\_of\_squares\_colon\_7\_full\_stop\_39 :Curve   [x]   (y)
\end{Verbatim}
            
    A common method to find such values \(a\) and \(b\) is the Least Squares
framework. Assume we have estimators \(\hat{a}\) and \(\hat{b}\) for
respectively \(a\) and \(b\). Then, we obtain estimators
\(\hat{y}_i, i \in \mathbb{N}\) for \(y_i\) by plugging in the
estimators \(\hat{a}\) and \(\hat{b}\) into the previous linear
equation, i.e. \[\hat{y}_i = \hat{a}x_i+\hat{b}.\]

If \(\hat{a}\) and \(\hat{b}\) are close to their optimal values, the
difference between \(y_i\) and \(\hat{y}_i\) should be small for each
\(i=1, \dots, n\). In the linear regression framework, to achieve this
goal we seek to minimize the sum of the squares of these differences and
thus obtain the following cost function
\[c(\hat{a},\hat{b}):= \sum_{i=1}^n (y_i - \hat{a}x_i+\hat{b})^2 = \sum_{i=1}^n (y_i - \hat{y}_i)^2.\]

This cost function for the example data set presented above can be
visualized in the following way:

    \begin{Verbatim}[commandchars=\\\{\}]
{\color{incolor}In [{\color{incolor}22}]:} \PY{n}{least\PYZus{}square\PYZus{}viz}
\end{Verbatim}


\begin{Verbatim}[commandchars=\\\{\}]
{\color{outcolor}Out[{\color{outcolor}22}]:} :Layout
            .DynamicMap.I  :DynamicMap   [a\_est,b\_est]
               :Overlay
                  .Polygons.I                                :Polygons   [x,y]
                  .Curve.Sum\_of\_squares\_colon\_7\_full\_stop\_39 :Curve   [x]   (y)
                  .Points.I                                  :Points   [x,y]
            .DynamicMap.II :DynamicMap   [a\_est,b\_est]
               :Overlay
                  .Image.I    :Image   [a estimate,b estimate]   (z)
                  .Contours.I :Contours   [a estimate,b estimate]   (z)
                  .Points.I   :Points   [x,y]
\end{Verbatim}
            
    On the left we visualize the data set together with the fitted slope (in
red) and the individual squared errors with respect to the slope
\(\hat{a}\) and the offset \(\hat{b}\). On the right we display the
contour levels corresponding to the cost function. The brighter an area
is, the lower the cost function of its encompassing points is. This can
be verified by watching the \texttt{sum\_of\_squares} value while
changing the values of \(\hat{a}\) and \(\hat{b}\). The combinations of
\(\hat{a}\) and \(\hat{b}\) are also represented by the red point on the
right plot.

    \subsection{4.4 Numerical Integration }\label{numerical-integration}

    Coming soon...

    \begin{Verbatim}[commandchars=\\\{\}]
{\color{incolor}In [{\color{incolor}3}]:} \PY{k+kn}{from} \PY{n+nn}{py\PYZus{}code}\PY{n+nn}{.}\PY{n+nn}{numerical\PYZus{}integration} \PY{k}{import} \PY{n}{int\PYZus{}approx\PYZus{}viz}
        \PY{n}{int\PYZus{}approx\PYZus{}viz}
\end{Verbatim}


    
    
    
    
    
    
\begin{Verbatim}[commandchars=\\\{\}]
{\color{outcolor}Out[{\color{outcolor}3}]:} :Layout
           .HoloMap.I  :HoloMap   [Steps]
              :Overlay
                 .NdOverlay.I :NdOverlay   [Element]
                    :Polygons   [x,y]
                 .Curve.I     :Curve   [x]   (f(x))
           .HoloMap.II :HoloMap   [Steps]
              :Overlay
                 .Curve.I  :Curve   [steps]   (error)
                 .Points.I :Points   [x,y]
                 .HLine.I  :HLine   [x,y]
\end{Verbatim}
            
    \subsection{Maybe one more?}\label{maybe-one-more}

\begin{itemize}
\tightlist
\item
  differential equation
\end{itemize}

    \section{5. Matrix Visualization}\label{matrix-visualization}

\subsection{5.1. Colormaps and Standard Matrices
}\label{colormaps-and-standard-matrices}

    In this chapter, we discuss a topic of matrix visualization. Common
practice to inspect a matrix seems to be to simply look through it in
raw format. This however is often a frustrating experience as the matrix
is usual too large to be completely displayed, may contain many digits
per entry and is usually far from being optimally formatted for viewing.
While extracting and rounding sub-matrices may be one option, it can be
informative to visualize the matrix as a 2D-colormap.

For this, we first briefly explore the topic of color mapping, i.e. the
process of translating numerical data into colors. We may for example be
interested to map low values to dark colors and high values to light
colors, as we have previously done Section \ref{repo_activity_viz} and
Section \ref{least_square_viz}. For this, we can use preconceived
palettes of colors - known as colormaps - that incorporate this
characteristic. The following visualization displays four continuous
sequential colormaps:

    \begin{Verbatim}[commandchars=\\\{\}]
{\color{incolor}In [{\color{incolor}20}]:} \PY{k+kn}{from} \PY{n+nn}{py\PYZus{}code}\PY{n+nn}{.}\PY{n+nn}{colormaps} \PY{k}{import} \PY{n}{cmap\PYZus{}seq\PYZus{}viz}\PY{p}{,} \PY{n}{cmap\PYZus{}div\PYZus{}viz}
         \PY{n}{cmap\PYZus{}seq\PYZus{}viz}
\end{Verbatim}


\begin{Verbatim}[commandchars=\\\{\}]
{\color{outcolor}Out[{\color{outcolor}20}]:} :Layout
            .Image.Blues   :Image   [x,y]   (z)
            .Image.Bmw     :Image   [x,y]   (z)
            .Image.Inferno :Image   [x,y]   (z)
            .Image.Viridis :Image   [x,y]   (z)
\end{Verbatim}
            
    In other circumstances, we may be interested in using diverging
colormaps, for example to distinguish between positive and negative
values. Four examples of colormaps with this characteristics are the
following:

    \begin{Verbatim}[commandchars=\\\{\}]
{\color{incolor}In [{\color{incolor}21}]:} \PY{n}{cmap\PYZus{}div\PYZus{}viz}
\end{Verbatim}


\begin{Verbatim}[commandchars=\\\{\}]
{\color{outcolor}Out[{\color{outcolor}21}]:} :Layout
            .Image.Bwr      :Image   [x,y]   (z)
            .Image.PiYG     :Image   [x,y]   (z)
            .Image.RdBu     :Image   [x,y]   (z)
            .Image.Spectral :Image   [x,y]   (z)
\end{Verbatim}
            
    Designing colormaps is surprisingly difficult. One important aspect is
to ensure that the luminance of the colormap varies smoothly across the
spectrum to help spot nuances in the data and enable informative black
and white displaying. Another aspect may be to enable accurate
representations for color blind viewers. It has been shown that the
traditionally used \texttt{\textquotesingle{}jet\textquotesingle{}}
colormap is especially ineffective at addressing these aspects and
encouraged the creation of the
\texttt{\textquotesingle{}Viridis\textquotesingle{}} project which
resulted in much improved colormaps. For more information on this topic,
we highly recommend the lecture
\href{https://www.youtube.com/watch?v=xAoljeRJ3lU}{"A Better Default
Colormap for Matplotlib"} by Nathaniel Smith.

    We now use the above colormaps to visualize some standard matrices: to
begin with, an identity matrix, a band block matrix and a triangular
matrix. All matrices are of dimension \(n\) x \(n\) with \(n=25\) and
the two used colormaps are
\texttt{\textquotesingle{}blues\textquotesingle{}} and
\texttt{\textquotesingle{}RdBu\textquotesingle{}}.

    \begin{Verbatim}[commandchars=\\\{\}]
{\color{incolor}In [{\color{incolor}1}]:} \PY{k+kn}{from} \PY{n+nn}{py\PYZus{}code}\PY{n+nn}{.}\PY{n+nn}{standard\PYZus{}matrices}\PY{n+nn}{.}\PY{n+nn}{py} \PY{k}{import} \PY{n}{standard\PYZus{}matrices\PYZus{}viz}\PY{p}{,} \PY{n}{corr\PYZus{}and\PYZus{}missing\PYZus{}viz}\PY{p}{,} \PY{n}{matrix\PYZus{}corr\PYZus{}big\PYZus{}viz}
        \PY{n}{standard\PYZus{}matrices\PYZus{}viz}
\end{Verbatim}


    
    
    
    
    
    
\begin{Verbatim}[commandchars=\\\{\}]
{\color{outcolor}Out[{\color{outcolor}1}]:} :Layout
           .Image.I   :Image   [x,y]   (z)
           .Image.II  :Image   [x,y]   (z)
           .Image.III :Image   [x,y]   (z)
\end{Verbatim}
            
    Note that we have again activated the hover-tool to enable looking at
individual values in the matrices.

Next, we generate two correlation matrices, one with strong dependence
structure (left) and one with weak dependence structure (right). The
third matrix represents a random matrix with \(100\) missing values
(green color). Here we set \(n=40\) and use the colormaps
\texttt{\textquotesingle{}bwr\textquotesingle{}} and again
\texttt{\textquotesingle{}blues\textquotesingle{}}.

    \begin{Verbatim}[commandchars=\\\{\}]
{\color{incolor}In [{\color{incolor}3}]:} \PY{n}{corr\PYZus{}and\PYZus{}missing\PYZus{}viz}
\end{Verbatim}


\begin{Verbatim}[commandchars=\\\{\}]
{\color{outcolor}Out[{\color{outcolor}3}]:} :Layout
           .Image.I   :Image   [x,y]   (z)
           .Image.II  :Image   [x,y]   (z)
           .Image.III :Image   [x,y]   (z)
\end{Verbatim}
            
    The correlation structure becomes immediately apparent upon
visualization, as do the missing values. Checking a matrix for missing
values can be particularly useful, as these often break many standard
functions.

Matrix visualization can also help us with larger data sets. To
demonstrate this, we generate a correlation matrix with strong
dependence structure for \(n=500\).

    \begin{Verbatim}[commandchars=\\\{\}]
{\color{incolor}In [{\color{incolor}4}]:} \PY{n}{matrix\PYZus{}corr\PYZus{}big\PYZus{}viz}
\end{Verbatim}


\begin{Verbatim}[commandchars=\\\{\}]
{\color{outcolor}Out[{\color{outcolor}4}]:} :Image   [x,y]   (z)
\end{Verbatim}
            
    Although each entry is now almost too small to be displayed, the
visualization can still help us to get a broad overview and the zoom
tool encourages to explore sub-matrices.

    \subsection{5.2. Recognizing Symmetric Matrices
}\label{recognizing-symmetric-matrices}

    Next we study how a matrix visualization could help us determine if a
matrix is symmetric or close to symmetric. Let us first visualize one
such matrix with \(n=40\).

    \begin{Verbatim}[commandchars=\\\{\}]
{\color{incolor}In [{\color{incolor}1}]:} \PY{k+kn}{from} \PY{n+nn}{py\PYZus{}code}\PY{n+nn}{.}\PY{n+nn}{symmetric\PYZus{}matrices} \PY{k}{import} \PY{n}{matrix\PYZus{}sym\PYZus{}viz}\PY{p}{,} \PY{n}{matrix\PYZus{}sym\PYZus{}examples\PYZus{}viz}\PY{p}{,} \PY{n}{diff\PYZus{}viz}\PY{p}{,} \PY{n}{diff\PYZus{}new\PYZus{}cmap\PYZus{}viz}\PY{p}{,} \PY{n}{ratio\PYZus{}viz}\PY{p}{,} \PY{n}{ratio\PYZus{}new\PYZus{}cmap\PYZus{}viz}
        \PY{n}{matrix\PYZus{}sym\PYZus{}viz}
\end{Verbatim}


    
    
    
    
    
    
\begin{Verbatim}[commandchars=\\\{\}]
{\color{outcolor}Out[{\color{outcolor}1}]:} :Image   [x,y]   (z)
\end{Verbatim}
            
    We are able to discern the symmetry, but if we were not told about it,
we would probably not have recognized it straight away. May there be
better ways to visualize this property? In real world applications, the
symmetry may sometimes not be one to one. Maybe there are just a few
values that differ to achieve perfect symmetry? Maybe there is some
noise added to the symmetry? If so, do we measure this noise on an
absolute basis or on a relative basis? Answering these questions
possibly require different visualizations. To demonstrate this, we
create three examples by modifying the above symmetric matrix. For the
first example, we add independent random normal variables (mean \(0\)
and variance \(0.15\)) to \(100\) values randomly selected, in the
second and third example we add respectively independent random normal
variables to all entries with mean \(0\) and variance \(0.02\) and
\(0.12\) respectively. The three matrices are the following.

    \begin{Verbatim}[commandchars=\\\{\}]
{\color{incolor}In [{\color{incolor}2}]:} \PY{n}{matrix\PYZus{}sym\PYZus{}examples\PYZus{}viz}
\end{Verbatim}


\begin{Verbatim}[commandchars=\\\{\}]
{\color{outcolor}Out[{\color{outcolor}2}]:} :Layout
           .Image.I   :Image   [x,y]   (z)
           .Image.II  :Image   [x,y]   (z)
           .Image.III :Image   [x,y]   (z)
\end{Verbatim}
            
    As discussed previously, it is difficult to deduce much from the
straight forward visualization of these three matrices. The, we begin by
visualizing the respective absolute differences between the lower and
upper triangular matrices and plot the results.

    \begin{Verbatim}[commandchars=\\\{\}]
{\color{incolor}In [{\color{incolor}3}]:} \PY{n}{diff\PYZus{}sym\PYZus{}examples\PYZus{}viz}
\end{Verbatim}


\begin{Verbatim}[commandchars=\\\{\}]
{\color{outcolor}Out[{\color{outcolor}3}]:} :Layout
           .Image.I   :Image   [x,y]   (z)
           .Image.II  :Image   [x,y]   (z)
           .Image.III :Image   [x,y]   (z)
\end{Verbatim}
            
    We now see with more clarity, what the differences between the two
triangular matrices are. While the colorbars on the right indicate the
ranges of the colormaps, from a pure visualization perspective, it may
make more sense to adapt the colors ranges to the value ranges of the
original matrices, minus diagionals (as they do not play a role in
determining symmetry). Thus we may adjust our plots in the following
way.

    \begin{Verbatim}[commandchars=\\\{\}]
{\color{incolor}In [{\color{incolor}4}]:} \PY{n}{diff\PYZus{}relative\PYZus{}sym\PYZus{}examples\PYZus{}viz}
\end{Verbatim}


\begin{Verbatim}[commandchars=\\\{\}]
{\color{outcolor}Out[{\color{outcolor}4}]:} :Layout
           .Image.I   :Image   [x,y]   (z)
           .Image.II  :Image   [x,y]   (z)
           .Image.III :Image   [x,y]   (z)
\end{Verbatim}
            
    This way, we see the relative difference between the low random noise
and heigh random noise examples more accurately. If absolute differences
are the measure with which we would define near symmetry, then from
these visualizations we would estimate that Example 1 and Example 3 do
not meet the criteria, while Example 2 does.

In the next plot we visualize the ratio between the lower triangular
matrix and the upper triangular matrix for each example. The number 1
has been subtracted so that the a value of zero (white color) represents
no change.

    \begin{Verbatim}[commandchars=\\\{\}]
{\color{incolor}In [{\color{incolor}2}]:} \PY{n}{ratio\PYZus{}viz}
\end{Verbatim}


\begin{Verbatim}[commandchars=\\\{\}]
{\color{outcolor}Out[{\color{outcolor}2}]:} :Layout
           .Image.I   :Image   [x,y]   (z)
           .Image.II  :Image   [x,y]   (z)
           .Image.III :Image   [x,y]   (z)
\end{Verbatim}
            
    Some ratios are dramatically high. This is due to some values in the
upper triangular matrix being near zero and thus leading to very large
ratios when dividing by them. This method is thus not ideal for our case
study. Nevertheless, we visualize this ratio a second time by changing
the range of the colormaps and thus clipping some of the very large and
very small values.

    \begin{Verbatim}[commandchars=\\\{\}]
{\color{incolor}In [{\color{incolor}3}]:} \PY{n}{ratio\PYZus{}new\PYZus{}cmap\PYZus{}viz}
\end{Verbatim}


\begin{Verbatim}[commandchars=\\\{\}]
{\color{outcolor}Out[{\color{outcolor}3}]:} :Layout
           .Image.I   :Image   [x,y]   (z)
           .Image.II  :Image   [x,y]   (z)
           .Image.III :Image   [x,y]   (z)
\end{Verbatim}
            
    In this visualization too, the same issue of near zero values lead to
large ratios for many matrix entries. Thus we discard this approach to
gauge the symmetry characteristics for these examples.

    \subsection{5.3. Gauss Elimination - LU Decomposition
}\label{gauss-elimination---lu-decomposition}

    Gauss elimination is the most important algorithm to solve a system of
linear equations, say \(Ax=y\) with \(A\) a \(n\) x \(n\) real matrix,
\(n \in \mathbb{N}\), \(x,y \in \mathbb{R}^n\). The idea is to transform
the system of equations into an equivalent one that is in the form of a
lower (\(L\)) and an upper triangular matrix (\(U\)). The reason is that
if \(L\) is a triangular matrix, the system \(Lx = y\) can easily be
solved by forward or backward substitution. I.e. if \(A = LU\), then
\(Ax=b \Longleftrightarrow L(Ux)=y \Longleftrightarrow Lz=y \text{ and } Ux=z\).

We now seek to visualize the Gauss elimination aglorithm without pivot.
For this, we define the vector
\(l^{(k)} = (0,\dots,t_{k+1},\dots, t_n)^T \in \mathbb{R}^n\) for
\(k \in \{1, \dots, n\}\) and the unit vector
\(e_k = (0, \dots, 0, 1, 0, \dots, 0)^T \in \mathbb{R}^n\). With this,
we define the Gauss-Matrix \(M_k\) in the following way

\[M_k := I_n-l^{(k)}e_k^T,\]

where \(I_n\) is the \(n\) x \(n\) identity matrix.

The characteristic of the Gauss-Matrix is best seen when left-multiplied
to the vector \(x:= (x_1, \dots, x_n)\):

\[M_kx = (x_1, \dots, x_k, 0, \dots, 0)^T.\]

Thus, with the help of these matrices, we can achieve the sought
decomposition with the following algorithm:

\textbf{Gauss elimination algorithm without pivot}

1. Initialization: \(A \in \mathbb{R}^{n \times n}\)

2. \(A^{(1)}:=A;\)

3. \textbf{for} \(k=1,\dots,n-1\)
\textbf{do}\(l^{(k)}:= \Bigg(0, \dots, 0, \frac{a_{k+1,k}^{(k)}}{a_{k,k}^{(k)}}, \dots, \frac{a_{n,k}^{(k)}}{a_{k,k}^{(k)}} \Bigg)\)
with \(k\) times \(0\) for the first entries;
\(M_k := I_n-l^{(k)}e_k^T;\) \(A^{(k+1)}:= M_kA^{(k)}\)\textbf{end}

Where \(U:= A^{(n)}\) is the resulting upper right triangular matrix,
\(L:= M_1^{-1} \dots M_{n-1}^{-1}\) is the lower left triangular matrix
and we have \(A=LU\).

We now apply the algorithm to a regular matrix
\(A \in \mathbb{R}^{15\times15}\) and visualize the three matrices for
each step.

    \begin{Verbatim}[commandchars=\\\{\}]
{\color{incolor}In [{\color{incolor}4}]:} \PY{k+kn}{from} \PY{n+nn}{py\PYZus{}code}\PY{n+nn}{.}\PY{n+nn}{gauss\PYZus{}elimination} \PY{k}{import} \PY{n}{gauss\PYZus{}elimination\PYZus{}viz}
        \PY{n}{gauss\PYZus{}elimination\PYZus{}viz}
\end{Verbatim}


    
    
    
    
    
    
\begin{Verbatim}[commandchars=\\\{\}]
{\color{outcolor}Out[{\color{outcolor}4}]:} :Layout
           .Image.I   :HoloMap   [Iteration]
              :Image   [x,y]   (z)
           .Image.II  :HoloMap   [Iteration]
              :Image   [x,y]   (z)
           .Image.III :HoloMap   [Iteration]
              :Image   [x,y]   (z)
\end{Verbatim}
            
    \subsection{5.4 Multidimentional Plotting with Color Channels
}\label{multidimentional-plotting-with-color-channels}

    In this last section, we study the possibility to plot multidimensional
images by superposing these using different color of the red-green-blue
(RGB) channels. For example, we may be interested in visualizing the
real part and the imaginary part of a complex matrix. For this, we could
color the real part in red and the imaginary part in blue. The library
\texttt{scipy} allows to generate specific matrices, among others a
complex discrete Fourier transform matrix. We use this matrix to
illustrate our example.

    \begin{Verbatim}[commandchars=\\\{\}]
{\color{incolor}In [{\color{incolor}5}]:} \PY{k+kn}{from} \PY{n+nn}{py\PYZus{}code}\PY{n+nn}{.}\PY{n+nn}{complex\PYZus{}matrix} \PY{k}{import} \PY{n}{complex\PYZus{}matrix\PYZus{}viz}
        \PY{n}{complex\PYZus{}matrix\PYZus{}viz}
\end{Verbatim}


    
    
    
    
    
    
\begin{Verbatim}[commandchars=\\\{\}]
{\color{outcolor}Out[{\color{outcolor}5}]:} :Layout
           .RGB.I   :RGB   [x,y]   (R,G,B)
           .RGB.II  :RGB   [x,y]   (R,G,B)
           .RGB.III :RGB   [x,y]   (R,G,B)
\end{Verbatim}
            
    The real and imaginary part of the complex matrix look rather similar,
in particular outside of the center. However, the superposition shows
that there are clearly differences and while most parts of both matrices
do overlap, they do not coincide. Thus, the blended version is much more
informative than the side by side comparison.

    \section{6. Summary and Outlook }\label{summary-and-outlook}

    \begin{itemize}
\tightlist
\item
  we have only looked at a fraction of the visualization space
\item
  3D visualization are particularly interesting and vast subject
\item
  in the end, imagination is the limit
\item
  more and more data is generated in almost every discipline
\item
  computers enable new ways of teaching:
  https://www.ted.com/talks/conrad\_wolfram\_teaching\_kids\_real\_math\_with\_computers
\end{itemize}

    \section{7. Bibliography }\label{bibliography}

    Formatting idea?

\begin{enumerate}
\def\labelenumi{\arabic{enumi}.}
\tightlist
\item
  Grant, Sanderson (2018) \emph{"3Blue1Brown YouTube Channel"} Available
  at: \url{https://www.youtube.com/channel/UCYO_jab_esuFRV4b17AJtAw}
  {[}Accessed: 31.08.2018{]}
\item
  Watt, Jeremy and Reza Borhani (2018) \emph{"Machine Learning Refined
  (Online Textbook)"} Available at
  \url{https://jermwatt.github.io/mlrefined/index.html} {[}Accessed:
  31.08.2018{]}
\end{enumerate}


    % Add a bibliography block to the postdoc
    
    
    
    \end{document}
